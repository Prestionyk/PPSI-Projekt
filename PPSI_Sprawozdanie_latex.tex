%%%%%%%%%%%%  Generated using docx2latex.com  %%%%%%%%%%%%%%

%%%%%%%%%%%%  v2.0.0-beta  %%%%%%%%%%%%%%

\documentclass[12pt]{article}
\usepackage{amsmath}
\usepackage{latexsym}
\usepackage{amsfonts}
\usepackage[normalem]{ulem}
\usepackage{soul}
\usepackage{array}
\usepackage{amssymb}
\usepackage{extarrows}
\usepackage{graphicx}
\usepackage[backend=biber,
style=numeric,
sorting=none,
isbn=false,
doi=false,
url=false,
]{biblatex}\addbibresource{bibliography-biblatex.bib}

\usepackage{subfig}
\usepackage{wrapfig}
\usepackage{wasysym}
\usepackage{enumitem}
\usepackage{adjustbox}
\usepackage{ragged2e}
\usepackage[svgnames,table]{xcolor}
\usepackage{tikz}
\usepackage{longtable}
\usepackage{changepage}
\usepackage{setspace}
\usepackage{hhline}
\usepackage{multicol}
\usepackage{tabto}
\usepackage{float}
\usepackage{multirow}
\usepackage{makecell}
\usepackage{fancyhdr}
\usepackage[toc,page]{appendix}
\usepackage[hidelinks]{hyperref}
\usetikzlibrary{shapes.symbols,shapes.geometric,shadows,arrows.meta}
\tikzset{>={Latex[width=1.5mm,length=2mm]}}
\usepackage{flowchart}\usepackage[paperheight=11.69in,paperwidth=8.27in,left=1.0in,right=1.0in,top=1.5in,bottom=1.5in,headheight=1in]{geometry}
\usepackage[utf8]{inputenc}
\usepackage[T1]{fontenc}
\TabPositions{0.5in,1.0in,1.5in,2.0in,2.5in,3.0in,3.5in,4.0in,4.5in,5.0in,5.5in,6.0in,}

\urlstyle{same}

\renewcommand{\_}{\kern-1.5pt\textunderscore\kern-1.5pt}

 %%%%%%%%%%%%  Set Depths for Sections  %%%%%%%%%%%%%%

% 1) Section
% 1.1) SubSection
% 1.1.1) SubSubSection
% 1.1.1.1) Paragraph
% 1.1.1.1.1) Subparagraph


\setcounter{tocdepth}{5}
\setcounter{secnumdepth}{5}


 %%%%%%%%%%%%  Set Depths for Nested Lists created by \begin{enumerate}  %%%%%%%%%%%%%%


\setlistdepth{9}
\renewlist{enumerate}{enumerate}{9}
		\setlist[enumerate,1]{label=\arabic*)}
		\setlist[enumerate,2]{label=\alph*)}
		\setlist[enumerate,3]{label=(\roman*)}
		\setlist[enumerate,4]{label=(\arabic*)}
		\setlist[enumerate,5]{label=(\Alph*)}
		\setlist[enumerate,6]{label=(\Roman*)}
		\setlist[enumerate,7]{label=\arabic*}
		\setlist[enumerate,8]{label=\alph*}
		\setlist[enumerate,9]{label=\roman*}

\renewlist{itemize}{itemize}{9}
		\setlist[itemize]{label=$\cdot$}
		\setlist[itemize,1]{label=\textbullet}
		\setlist[itemize,2]{label=$\circ$}
		\setlist[itemize,3]{label=$\ast$}
		\setlist[itemize,4]{label=$\dagger$}
		\setlist[itemize,5]{label=$\triangleright$}
		\setlist[itemize,6]{label=$\bigstar$}
		\setlist[itemize,7]{label=$\blacklozenge$}
		\setlist[itemize,8]{label=$\prime$}

\setlength{\topsep}{0pt}\setlength{\parindent}{0pt}

 %%%%%%%%%%%%  This sets linespacing (verticle gap between Lines) Default=1 %%%%%%%%%%%%%%


\renewcommand{\arraystretch}{1.3}


%%%%%%%%%%%%%%%%%%%% Document code starts here %%%%%%%%%%%%%%%%%%%%



\begin{document}
\section*{PPSI}
\addcontentsline{toc}{section}{PPSI}
\setlength{\parskip}{12.0pt}
\textbf{Członkowie}:
\setlength{\parskip}{0.0pt}
\begin{itemize}
	\item \href{https://github.com/PiotrTekieli}{\textcolor[HTML]{1155CC}{\ul{Piotr Tekieli}}}\  
	\item \href{https://github.com/sam21401}{\textcolor[HTML]{1155CC}{\ul{Samuel Leonczyk}}}\  
\setlength{\parskip}{12.0pt}
	\item \href{https://github.com/Prestionyk}{\textcolor[HTML]{1155CC}{\ul{Mariusz Skuza}}}\  
\end{itemize}

\vspace{\baselineskip}
{\fontsize{18pt}{21.6pt}\selectfont \textbf{Projekt aplikacji forum internetowe}}

\vspace{\baselineskip}
Spis treści
\setlength{\parskip}{0.0pt}
\begin{enumerate}
	\item Opis funkcjonalny systemu.
	\item Wyszczególnione wdrożone kwalifikacje.
	\item Streszczenie opisu technologicznego.
	\item Instrukcja lokalnego uruchomienia systemu.
\setlength{\parskip}{12.0pt}
	\item Wnioski projektowe.
\end{enumerate}
\textbf{1. Opis funkcjonalny systemu:} 
Jest to aplikacja internetowa w postaci forum, która pozwala tworzyć wątki, edytować je i usuwać oraz dodawać do nich komentarze.

\vspace{\baselineskip}
\textbf{2. Wyszczególnione wdrożone kwalifikacje:}
\setlength{\parskip}{0.0pt}
\begin{enumerate}
	\item HTML5 - jest jako niezbędny element do wyświetlania treści na stronie.
	\item CSS3 - formatowanie treści za pomocą bootstrap 4.
	\item Formularze - wykorzystane do tworzenia użytkownika, komentarzy oraz samych postów.
	\item Baza danych - połączenie z lokalną bazą danych, w której przechowywane są użytkownicy, komentarze oraz posty.
	\item Router - wykorzystanie RouteAttribiute to zarządzania ścieżkami dostępu.
	\item Uwierzytelnianie - rejestracja i logowanie uztkownikó za pomocą ASP.NET Identity.
	\item MVC - wykorzystanie wzorca mvc w celu łatwiejszej lokalizacji elementów projektu systemu internetowego poprzez dzielenie aplikacji na wyspecjalizowane części.
	\item CRUD - tworzenie komentarzy, postów oraz ich edycja, usuwanie i przeglądanie. \tab 
	\item ORM - wykorzystanie Package Manager Console do zmigrowania modeli systemu do bazy danych oraz używanie programowania obiektowego do manipulowania danych z bazy danych.
	\item Wystawianie API - umożliwiamy pobranie danych o wątkach i komentarzach.\tab 
	\item Konsumowanie API - jest możliwość wybrania miasta aby sprawdzić jego aktualną temperaturę przy wykorzystaniu API z openweathermap.
	\item AJAX - asynchroniczne zapytania pozwalają na szybkie wyświetlanie nowych \tab stron komentarzy bez konieczności odświeżania całej strony.
	\item Mail - Możliwość wysłania maila przez protokół SMTP wykorzystując \tab skonfigurowaną pocztę gmail.
	\item Lokalizacja - brak.
	\item RWD - strona poprawnie wyświetla się na wielu różnych rozdzielczościach.
	\item Logger - prosty logger z biblioteki Microsoft.Extensions.Logging, który pozwala monitorować aktywność na naszej stronie.
	\item Cache \tab - informacje o wątkach będą pobierane najczęściej co 5 sekund, ponieważ serwer je zapisuje, aby zmniejszyć obciążenie na serwerze.
	\item System zarządzania zależnościami - brak.
	\item Automatyzacja - brak.
\setlength{\parskip}{12.0pt}
	\item SEO\ - poprawne tytuły, słowa kluczowe i opisy dla stron w formie meta tagów, aby pomóc wyszukiwarkom poprawnie znaleźć i opisać strony.  
\end{enumerate}
\textbf{3. Streszczenie opisu technologicznego}
Frontend i backend jest oparty na frameworku ASP.NET MVC w wersji 5. Pliki wyświetlające treść na stronie są zapisane za pomocą Razor, który jest połączeniem języków C$\#$  i HTML. Pozwala na wykorzystywanie metod i funkcji C$\#$  przy renderowaniu strony HTML. Dodatkowo frontend używa biblioteki CSS Bootstrap.

\vspace{\baselineskip}

\vspace{\baselineskip}

\vspace{\baselineskip}

\vspace{\baselineskip}
\textbf{4. Instrukcje lokalnego uruchomienia systemu}
Aby móc korzystać z aplikacji, należy sklonować repozytorium. Następnie otworzyć projekt w Visual Studio. Aby aplikacja działała poprawnie, należy mieć zainstalowany dodatek:


%%%%%%%%%%%%%%%%%%%% Figure/Image No: 1 starts here %%%%%%%%%%%%%%%%%%%%

\begin{figure}[H]
	\begin{Center}
		\includegraphics[width=4.24in,height=1.05in]{./media/image2.png}
	\end{Center}
\end{figure}


%%%%%%%%%%%%%%%%%%%% Figure/Image No: 1 Ends here %%%%%%%%%%%%%%%%%%%%


\vspace{\baselineskip}oraz NET 5.0 SDK ze strony Microsoft\href{https://dotnet.microsoft.com/download}{ }\href{https://dotnet.microsoft.com/download}{\textcolor[HTML]{1155CC}{\ul{https://dotnet.microsoft.com/download}}} Następnie przechodzimy do Package Manager Console i komendą "add-migration [nazwa]" oraz "update-database" i uruchamiamy projekt.

\vspace{\baselineskip}
\textbf{5. Podsumowanie/wnioski projektu}
Projekt został wykonany we frameworku ASP.NET MVC. Początki były trudne, ponieważ nigdy wcześniej nie mieliśmy kontaktu z takim zadaniem, ale z biegiem czasu praca stawała się coraz przyjemniejsza. Framework ma bardzo dużo wbudowanych funkcjonalności, które przyśpieszają pracę. Wcześniej nie mieliśmy również kontaktu z bibliotekami CSS takimi jak bootstrap. Bootstrap okazał się bardzo przydatny i przyjemny w użyciu. Poniżej przykładowy ss z aplikacji:


%%%%%%%%%%%%%%%%%%%% Figure/Image No: 2 starts here %%%%%%%%%%%%%%%%%%%%

\begin{figure}[H]
	\begin{Center}
		\includegraphics[width=6.27in,height=3.53in]{./media/image1.png}
	\end{Center}
\end{figure}


%%%%%%%%%%%%%%%%%%%% Figure/Image No: 2 Ends here %%%%%%%%%%%%%%%%%%%%


\vspace{\baselineskip}\printbibliography
\end{document}